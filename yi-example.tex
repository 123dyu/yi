\documentclass{ctexart}

\usepackage{yi,
            tikz,
           }

\title{\textsf{yi}宏包与\textsf{tikz}宏包结合使用案例}
\author{dyu}
\date{\today}

\begin{document}

\maketitle

\section{绘制先天八卦次序图}

《系辞传》曰:“易有太极,是生两仪,两仪生四象,四象生八卦。”

\begin{figure}[h]
  \centering
  \ExplSyntaxOn
  \begin{tikzpicture}
    \int_step_inline:nnn {1} {8}
      { 
        % 放置先天八卦数
        \node ( a #1 ) at ( #1 , 7 ) {#1}; 
        \clist_clear:N \l_tmpa_clist
        % 存储先天八卦名
        \clist_set:Nn \l_tmpa_clist { 乾, 兑, 离, 震, 巽, 坎, 艮, 坤 }
        % 放置先天八卦名
        \node ( b #1 ) at ( #1 , 6.5 ) 
          { \clist_item:Nn \l_tmpa_clist {#1} };
        % 放置先天八卦
        \node ( c #1 ) at ( #1 , 6   ) 
          { \yao[ jinzhi=dec, type=bagua ]{ \int_eval:n { 8-#1 } } };
      }
    \int_step_inline:nnn {1} {4}
      {
        % 放置四象
        \node ( d #1 ) at ( 2*#1-0.5 , 5 ) 
          { \yao[ jinzhi=dec, type=sixiang ]{ \int_eval:n { 4-#1 } } };
        \clist_clear:N \l_tmpa_clist
        % 存储四象名
        \clist_set:Nn \l_tmpa_clist { 太阳, 少阴, 少阳, 太阴 }
        % 放置四象名
        \node () at ( 2*#1-0.5 , 4.6 ) 
          { \kaishu \clist_item:Nn \l_tmpa_clist {#1} };
      }
    \int_step_inline:nnn {1} {2}
      {
        % 放置两仪
        \node ( e #1 ) at ( #1+#1^2+0.5 , 4 ) 
          { \yao[ jinzhi=dec, type=liangyi ]{ \int_eval:n { 2 - #1 } } };
        \clist_clear:N \l_tmpa_clist
        % 存储两仪名
        \clist_set:Nn \l_tmpa_clist { 阳爻, 阴爻 }
        % 放置两仪名
        \node () at ( #1+#1^2+0.5 , 3.6 ) 
          { \kaishu \clist_item:Nn \l_tmpa_clist {#1} };
      }
    % 放置太极图
    \node (f1) at ( 4.5 , 3   ) { \taiji[scale=1.6] };
    % 放置“太极”二字
    \node ()   at ( 4.5 , 2.4 ) { \fbox{太极} };
    % 太极图向两仪连线
    \draw ( f1 ) -- ( e1 );
    \draw ( f1 ) -- ( e2 );
    % 两仪向四象连线
    \int_step_inline:nnn {1} {2}
      {
        \draw ( e #1 ) -- ( d \int_eval:n { #1*2 - 1 } );
        \draw ( e #1 ) -- ( d \int_eval:n { #1*2 } );
      }
    % 四象向八卦连线
    \int_step_inline:nnn {1} {4}
      {
        \draw ( d #1 ) -- ( c \int_eval:n { #1*2 - 1 } );
        \draw ( d #1 ) -- ( c \int_eval:n { #1*2 } );
      }
  \end{tikzpicture}
  \ExplSyntaxOff
\end{figure}



\section{绘制先天太极八卦图}

《说卦传》曰:“天地定位,山泽通气,雷风相薄,水火不相射,八卦相错,数往者顺,知来者逆。”

\begin{figure}[h]
    \centering
    \ExplSyntaxOn
    \begin{tikzpicture}
      % 将太极图放置于圆心
      \node (a) at ( 0 , 0 ) { \taiji[scale=5,reflect] };
      % 逆时针画乾、兑、离、震四卦
      \int_step_inline:nnn {1} {4}
        {
          \node [ rotate = 45*(#1-1) ] () at 
            ( \int_eval:n { ( #1-1 )*45 + 90 } \c_colon_str 2  )
            { 
              \yao 
                [ jinzhi=dec, type=bagua, scale=3 ]
                { \int_eval:n{ #1+2*(4-#1) } }
            };
        }
      % 顺时针画巽、坎、艮、坤四卦
      \int_step_inline:nnn {1} {4}
        {
          \node [ rotate = -45*#1 ] () at 
            ( \int_eval:n { 90 - #1*45 } \c_colon_str 2) 
            { 
              \yao
                [ jinzhi=dec, type=bagua, scale=3 ]
                { \int_eval:n { 4-#1 } } 
            };
        }
      \clist_clear:N \l_tmpa_clist
      \clist_clear:N \l_tmpb_clist  
      % 存储先天八卦名和先天八卦数
      \clist_set:Nn \l_tmpa_clist 
        { 乾(一), 兑(二), 离(三), 震(四), 坤(八), 艮(七), 坎(六), 巽(五) }
      % 存储方位
      \clist_set:Nn \l_tmpb_clist 
        { 南方, 东南, 东方, 东北, 北方, 西北, 西方, 西南 }
      \int_step_inline:nnn {1} {8}
        {
          % 放置先天八卦名和先天八卦数
          \node [ rotate = (#1-1)*45 ] () at 
            ( \int_eval:n { (#1-1)*45 + 90 } \c_colon_str 2.6 ) 
            { \bfseries \clist_item:Nn \l_tmpa_clist {#1} };
          % 放置方位
          \node [ rotate = (#1-1)*45 ] () at 
            ( \int_eval:n{(#1-1)*45+90} \c_colon_str 3.2) 
            { \itshape \clist_item:Nn \l_tmpb_clist {#1} };
        }
    \end{tikzpicture}
    \ExplSyntaxOff
    \caption{先天太极八卦图}
\end{figure}



\section{绘制后天太极八卦图}

《说卦传》:“帝出乎震,齐乎巽,相间乎离,致役乎坤,说言乎兑,战乎乾,劳乎坎,成言乎艮。”

\begin{figure}
  \centering
  \ExplSyntaxOn
  \begin{tikzpicture}
    % 将太极图放置于圆心
    \node (a) at ( 0 , 0 ) { \taiji[scale=5] };
    % 新建四个逗号分隔列表变量,用于存储八卦、八卦名、方位、五行
    \clist_clear_new:N \l_houtian_tmpa_clist
    \clist_clear_new:N \l_houtian_tmpb_clist
    \clist_clear_new:N \l_houtian_tmpc_clist
    \clist_clear_new:N \l_houtian_tmpd_clist
    % 将八卦放大三倍
    \yisetup{ yao/scale=3 }
    % 存储八卦
    \clist_set:Nn \l_houtian_tmpa_clist 
      {
        \yao{101}, 
        \yao{011},
        \yao{100}, 
        \yao{001}, 
        \yao{010}, 
        \yao{111}, 
        \yao{110}, 
        \yao{000} 
      }
    % 自离卦开始逆时针存储八卦名
    \clist_set:Nn \l_houtian_tmpb_clist 
      { 离, 巽, 震, 艮, 坎, 乾, 兑, 坤 }
    % 自南方开始逆时针存储八卦方位
    \clist_set:Nn \l_houtian_tmpc_clist 
      { 南方, 东南, 东方, 东北, 北方, 西北, 西方, 西南 }
    % 自火属性开始逆时针存储五行
    \clist_set:Nn \l_houtian_tmpd_clist 
      { 火, 木, 木, 土, 水, 金, 金, 土 }
    \int_step_inline:nnn {1} {8}
      {
        % 放置八卦
        \node [ rotate = (#1-1)*45 ] () at 
          ( \int_eval:n { (#1-1)*45 + 90 } \c_colon_str 2 ) 
          { \clist_item:Nn \l_houtian_tmpa_clist {#1} };
        % 放置八卦名
        \node [ rotate = (#1-1)*45 ] () at
          ( \int_eval:n { (#1-1)*45 + 90 } \c_colon_str 2.6 ) 
          { \bfseries \clist_item:Nn \l_houtian_tmpb_clist {#1} };
        % 放置八卦方位
        \node [ rotate = (#1-1)*45 ] () at 
          ( \int_eval:n { (#1-1)*45 + 90 } \c_colon_str 3.2 ) 
          { \itshape \color{purple} \clist_item:Nn \l_houtian_tmpc_clist {#1} };
        % 放置八卦五行
        \node [ rotate = (#1-1)*45 ] () at 
          ( \int_eval:n { (#1-1)*45 + 90 } \c_colon_str 3.8 ) 
          { \itshape \clist_item:Nn \l_houtian_tmpd_clist {#1} };
      }
  \end{tikzpicture}
  \ExplSyntaxOff
  \caption{后天太极八卦图}
\end{figure}



\end{document}